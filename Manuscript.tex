\documentclass{QHUMaster}
\usepackage{graphicx}
\usepackage{newtxtext}
\usepackage{lipsum}
\usepackage{zhlipsum}



\begin{document}
%%% 中文封面内容
\renewcommand{\title}{The sunset is like a rustling duck, and the autumn water is like a long sky} % 英文标题
\renewcommand{\biaoti}{落霞与孤鹜齐飞秋水共长天一色}  % 中文标题
\renewcommand{\xueyuan}{机械工程学院}
\renewcommand{\zhuanye}{机械电子工程}
\renewcommand{\xingming}{coffeelze}
\renewcommand{\xuehao}{YS210802020148}
\renewcommand{\daoshi}{Coffeelze}
\renewcommand{\grade}{2021}
\renewcommand{\research}{滕王阁诗序}
\renewcommand{\dateOfGrant}{2021年5月20日}
\renewcommand{\startAndFinish}{2020.11-2023.03}

%%% 英文封面填写内容
\renewcommand{\enMentorName}{Coffeelize}
\renewcommand{\enAuthor}{Zhipeng Wu}
\renewcommand{\enMajor}{Master of Engineering}
\renewcommand{\enSpecificMajor}{Mechanical and Electronic Engineering}
\renewcommand{\enDepartment}{College of Mechanical Engineering}
\renewcommand{\enStartAndFinish}{November, 2020—March, 2023}



%%% 中文封面
\titlepage
%%% 英文封面
\enTitlepage

%%%
%前言部分
%%%
\frontmatter

%%% 中文摘要
\begin{zhAbsract}{关键词1;关键词2}
	\zhlipsum[1]
\end{zhAbsract}
%%% 英文摘要
\begin{enAbsract}{keyword1; keyword2}
	\lipsum[1]
\end{enAbsract}

%%% 目录
\tableofcontents
\addcontentsline{toc}{chapter}{目\quad 录}
\clearpage
\setcounter{page}{1}


%%%
%文章主体
%文章结构请勿删除,配合相应的页面样式使用
%%%
\mainmatter

\songti\zihao{-4} 

\chapter{时运不齐,命途多舛}
\section{冯唐易老,李广难封}
\zhlipsum[1-3]
\section{老当益壮,宁移白首之心}
\zhlipsum[1-5]
\section{穷且益坚,不坠青云之志}
\subsection{三尺微命,一介书生}
这是参考论文引用样式\cite{kocher99,cnproceed}

\zhlipsum[1-3]
\chapter{豫章故郡,洪都新府}
\section{千里逢迎}
\zhlipsum[1-3]
\section{高朋满座}
\zhlipsum[1-5]
\section{腾蛟起凤,孟学士之词宗}
\chapter{时维九月,序属三秋}
\zhlipsum[1-3]
\section{潦水尽而寒潭清}
\zhlipsum[1-5]
\section{烟光凝而暮山紫}
\chapter{遥襟甫畅,逸兴遄飞}
\zhlipsum[1-3]
\section{落霞与孤鹜齐飞,秋水共长天一色}
\zhlipsum[1-5]

%%%
%论文后部
%%%
\backmatter
%=======
%打印参考文献
%=======
\bibdatabase{bib/reference}
\printbib

%\chapter*{附\quad 录}\addcontentsline{toc}{chapter}{附\quad 录}
\chapter{附\quad 录}
是正文主体的补充项目,并不是必需的。下列内容可以作为附录:
(1)为了整篇材料的完整,插入正文又有损于编排条理性和逻辑性的材料;
(2)由于篇幅过大,或取材于复制件不便编入正文的材料;
(3)对一般读者并非必须阅读,但对本专业人员有参考价值的资料;
(如外文文献复印件及中文译文、公式的推导、程序流程图、图纸、数据表格等)
附录按“附录A,附录B,附录A1“等编号。
请单击样式“附录1”为第1级的附录编号,样式“附录2”为第二级的附录编号,样式“附录3”控制第三级别的样式。

\chapter{致\quad 谢}

感谢我的导师XXX老师,谢谢他对我的悉心指导。
他无私的关爱和严谨的治学态度,将激励我不断的进取,走好以后的道路。
其次,还要感谢在这四年的学习中教过我的所有老师们,谢谢他们传授给了我知识。
我的同学XXX,在写作的过程中给我提供了一些宝贵的资料和建议,在此一并感谢!

%%%
% 作者在读期间科研成果简介
%%%
\Achievements
一、发表论文情况

1.Article here ...

\blank

二、参与课题情况

三、获奖情况

\end{document}